1General Purpose
1.1Mandatory Criteria

The following requirements must be fulfilled. Each is derived directly from the User Requirements Specification.

The system shall select and formalize a representative structured risk-assessment method (e.g., HAZOP, FMECA) to serve as the benchmark.

The system shall evaluate multiple state-of-the-art, off-the-shelf LLMs on the selected benchmark to establish baseline performance.

The system shall integrate a retrieval pipeline that supplies domain-specific risk-assessment knowledge to LLMs and enables re-evaluation under RAG conditions.

The system shall perform a comparative analysis of baseline and RAG-enhanced evaluations, using both quantitative metrics and qualitative assessments.

1.2Optional Criteria

The following requirement may be implemented if time allows:

The system may fine-tune at least one LLM on curated risk-assessment corpora and evaluate its performance using the benchmark.

1.3Demarcation Criteria

The following aspects are explicitly excluded to keep the project scope reasonably confined:

The system will not develop an automated industrial risk-assessment tool.

The system will not integrate with real industrial systems and safety-critical environments.

The system will not use confidential and restricted industrial data.

Real-time performance, deployment, or operational integration is not included.

No new LLM models will be created, only existing off-the-shelf models will be evaluated.

Model outputs will not be used to make real safety-critical decisions.

The project focuses solely on research experiments, not the development of a practical industrial tool.
2Operational Area
2.1Areas of Application
The system is intended for academic research purposes within the scope of this research project. 
Its primary applications include:
Creating a structured evaluation benchmark for LLMs in risk-assessment contexts.

Running baseline and RAG-enhanced LLM experiments.

Supporting fine-tuning experiments (optional).

Providing insights into the feasibility of LLM-based support in structured engineering analysis.
2.2User Group
The main users are:

The research project author - Munazzar Ajreen.

The supervisor and examiner (reviewers of results).

Future researchers who may reuse or extend the system framework
3Requirements to the Conception
All requirements from the user requirements specification must be referenced.

/CR10/	Functional Requirements


/CR11/	Benchmark: The research thesis shall select and formalize a representative structured risk assessment method (e.g., HAZOP, FMECA) to be used as a benchmark.

/CR12/	Assessment Criteria: The research thesis shall derive relevant assessment criteria that allow systematic comparison of LLM performance against the selected risk assessment benchmark.

/CR13/	LLM: The research thesis shall evaluate multiple state-of-the-art, off-the-shelf LLMs on the chosen benchmark to establish baseline performance.

/CR14/
	RAG: The research thesis shall integrate a retrieval pipeline that supplies domain-specific risk assessment knowledge to LLMs and enable re-evaluation of performance against the baseline.

/CR15/	
Fine-Tuning (Optional - If time allows): The research thesis may fine-tune at least one LLM on curated risk assessment corpora and evaluate its performance on the benchmark.

/CR16/	Evaluation: The research thesis shall perform a comparative analysis of baseline and RAG-enhanced, including quantitative metrics and qualitative assessments.

/CR17/	Report: The research thesis shall document results in a structured format and provide insights into the applicability of LLMs to real-world risk assessment practice.

	



/CR20/	Non-Functional Requirements

Implementation:

/CR21/	The research thesis shall implement the solution in Python. 
	
/CR22/	The research thesis shall design the solution using object-oriented programming principles to ensure modularity, clarity, and extensibility.

/CR23/
	The research thesis shall maintain the full codebase in a Git repository with version history, branching, and adequate documentation.

/CR24/	The research thesis shall ensure the reproducibility of experiments by systematically logging and storing prompts, outputs, datasets, and evaluation metrics.

/CR25/	
The research thesis shall report model types, configurations, hyperparameters, and data sources used in all experiments in a transparent manner.

4Quality Requirements
Product Quality	very high	high	normal	not relevant
Theory		x		
Universality			x	
Consistency			x	
Functionality			x	
Correctness			x	
Applicability			x	
Safety			x	
Usability			x	
Comprehensibility			x	
Learnability			x	
Adaptability			x	
Analyzability			x	
Modifiability			x	

Terminology of Quality Requirements:
1.Theory: Presence of scientific methods of approach
Universality: Degree to which the theoretical approach is generalizable
Consistency: Degree to which the theoretical approach is non-contradictory and cohesive

2.Functionality: Presence of features with specified characteristics meeting pre-defined requirements
Correctness: Providing of correct results and outputs as specified before 
Applicability: Degree to which the concept shall be practically implemented

3.Safety (according to VDI 3542/1): State in which the actual risk is smaller than the maximum justifiable risk (absence of danger). Is maintained by meeting correctness requirements.

4.Usability: Effort necessary to allow use and individual assessment of said effort by pre-defined user groups.
Comprehensibility: Effort necessary to understand the concept
Learnability: Effort necessary to learn the concept (e.g regarding handling, maintaining, inputs or outputs)

5.Adaptability: Effort necessary to implement changes. Changes can be corrections, improvements or adaptions to changes of the environment, the requirements or functional specifications.
Analyzability: Effort to diagnose faults or the causes of defects or to identify parts to be changed
Modifiability: Effort to carry out improvements, correct faults or adapt to external changes
5Execution
The thesis has to be executed according to the “IAS Process Model” (Model for Conceptional Projects). 

The current state of the thesis and results have to be discussed with the tutor every two weeks.

The IAS guidelines have to be respected.

6. Additions
None.
7 Literature
None.	

